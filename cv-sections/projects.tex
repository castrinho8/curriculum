%----------------------------------------------------------------------------------------
%	SECTION TITLE
%----------------------------------------------------------------------------------------

\cvsection{Proyectos personales}

%----------------------------------------------------------------------------------------
%	SECTION CONTENT
%----------------------------------------------------------------------------------------

\begin{cventries}

%------------------------------------------------

\cventry
{Proyecto fin de carrera} % Institution
{\href{https://github.com/vacmatch/vacmatch-mobile}{VACmatch Mobile}} % Degree
{A Coruña} % Location
{Dic. 2015 - Actualidad} % Date(s)
{ % Description(s) bullet points
\begin{cvitems}
\item {Aplicación para gestión de actas deportivas que permite a los 
árbitros mantener los resultados actualizados en tiempo real en la web de una 
federación deportiva e incluso funcionar offline cuando no hay cobertura.}
\item {Es una aplicación móvil híbrida creada con React, con funcionamiento 
offline, utilizando bases de datos NoSQL (PouchDB y CouchDB), distribuida con 
una imagen de Docker y bajo integración continua con Travis.}
\item {Se ha realizado Test Driven Development durante la segunda mitad del 
desarrollo, testeando la lógica de negocio de los servicios y utilizando la 
librería Jest, basada en Jasmine.}
\end{cvitems}
}

\cventry
{Proyecto de GPUL} % Institution
{\href{https://github.com/gpul-org/XEA}{XEA}} % Degree
{A Coruña} % Location
{Jun. 2016 - Actualidad} % Date(s)
{ % Description(s) bullet points
\begin{cvitems}
\item {Aplicación web para gestión colaborativa de espacios, eventos y 
entradas, actualmente en los primeros pasos de su desarrollo.}
\item {Backend formado por una API REST utilizando Django, Django REST 
Framework y realizando Test Driven Development.}
\item {Frontend creando una Single Page Application con React y Redux.}
\end{cvitems}
}


\end{cventries}